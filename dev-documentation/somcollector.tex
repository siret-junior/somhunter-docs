\chapter{SOMCollector}
\label{somcollector}

In this chapter, we will describe SOMCollector. This software was created for the user interaction logs collection and thus its possibilities are reduced in comparison to SOMHunter. In the following sections, we will describe the differences between SOMCollector and SOMHunter and what was added. This project has additional material in Czech like the user manual and data description on the wiki\footnote{https://github.com/siret-junior/som-collector/wiki}. 

\section{HTTP and Core API}
In this version, the HTTP API is managed by the Node.js layer and the core API is called through the Node.js module system. The functionality of the core API is smaller than in SOMHunter because it was forked from a version that did not support that many features. The additions to the SOMHunter were three: multi-user support, a new specific type of logging, and a simplified search flow. The multi-user support is provided by \lstinline{class SomHuntersGuild} which holds SOMHunter objects for each user. The specific logging is implemented in \lstinline{FeedbackLogger} where it creates for each rescore call a single line CSV file. The logged format of the data is described in the wiki of the project\footnote{https://github.com/siret-junior/som-collector/wiki -> Příručka k logovaným datům}. After the data collection, these files can be concatenated and the data analysis can be performed. The simplified version of the search flow was implemented mainly in the UI. In the core we added a target images sequence, that was read from a given file. The current target image was searched by a user and logged in the CSV file.

\section{UI}

The user interface is written in HTML and vanilla JS. The original SOMHunter UI from the older version was simplified only to enter a text query and show 64 frames. Additional features like gaining experience and levelling up were introduced to motivate users to search for a longer time. This was implemented mainly in the UI in function \lstinline{showSubmitWindow}.

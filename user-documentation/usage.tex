\section{Usage}
\label{sec:usage}


In this chapter, we'll describe the typical workflow of how we approached the tasks during the competitions. 

The first thing we try is to describe the task using natural language, ideally using both first and second temporal query input. We also apply the filters if they are applicable. Then we try both textual models (default and buttons with the "V"). We then browse a decent amount of the top results while looking for good examples to like. If top-scored results are too similar and we need a broader view we go to SOM display. It may also be worth checking the relocation display if there are some good example images that may be more precise than the text query.

Every time we find around two or three likes, we ask for a re-score. Then we repeat the process if we feel that we're moving in the right direction.
Whenever we need to inspect more of the temporal context of the frame, we use the replay feature (shift + mouse wheel). If we need more we go to the detail window.

This is the usual workflow that leads to finding the task. Of course, if you know the positional relations of the frame object, you should try to sketch them using the canvas queries. Also, if you are unable to describe the task with a natural language, but have a good bitmap describing it, use it.
